
This thesis aimed at addressing some of the challenges that are still
at stake for a measurement of the \Gls{CP} violating phase using
neutrino oscillations. The \Gls{NCg} cross section is already one the
largest unknown in the measurement of the appearance signal at the far
detector.

The search for \Gls{NCg} leads to a limit of $\dataresult$ at
$90\%$~\Gls{CL} for these processes on carbon. At the time this
conclusion was written, the \Gls{NCg} was already the biggest cross
section error for the electron sample in \Gls{RHC} at \Gls{SK}. Of
course, the poor statistical power of this sample and the \Gls{SK}
detector uncertainties are still, by far, the main uncertainties, but
both of these are expected to improve in the future. Within the
current paradigm for the next generation of near detectors, it should
be feared that nobody will be able to measure this cross section on a
light isoscalar target in time to be able to characterise this process
in time for \Gls{HK}, and certainly not using the \Gls{TK} flux.

Similarly, the electron neutrino cross section is a fundamental input
for a \Gls{CP} violation measurement in the neutrino sector. To date,
there is no published, exclusive measurement of the anti-electron cross
section and the electron neutrino equivalents suffers poor statistics
and unexplained (or uncontrolled) backgrounds. There is currently no
other constraints used in the \Gls{TK} oscillation analyses than a
flat constrain on the electron (anti-) neutrino, which is still very
far from reach experimentally.

Both of these issues described in this thesis are complicated
experimental problems. It seems that the way of appropriately dealing
with this problem is to use a large homogeneous detector that can
efficiently reject external photons and differentiate them from single
electrons. This probably requires a large and heavy scintillator
target with a small granularity, submerged in a magnetic field and
exposed to the \Gls{TK} off-axis flux.