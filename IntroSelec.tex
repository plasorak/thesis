

This chapter details the selection of the \Gls{NCg} events. First, the
data sets are described. This section broadly explains how the
triggers, calibration, reconstruction work and show which samples were
used for the analysis. Then, the second section details the selection
cuts. Finally, the performance of selection is shown.

The \Gls{NCg} selection is largely based on the so-called ``gamma
selection'', developed as a control sample for the \Gls{nue} \Gls{CC}
inclusive cross section as was used in~\cite{nueT2K}.

In short, it relies on identifying two electron tracks of opposite
charges in the \Gls{TPC} that come from the \Gls{FGD}1. To be sure
that these tracks come from a photon conversion, some simple
requirements are made on the reconstructed invariant mass and the
distance between the two tracks. Some vetoes are also added to reduce
the contamination coming from outside the fiducial volume (\Gls{OOFV})
and from \Gls{numu} \Gls{CC} interactions.

% The selected topology is illustrated in
% Figure~\ref{fig:eventdisplay}, where event displays of a selected
% \Gls{NCg} events (\Gls{MC} and data) are shown.

% \begin{figure}[ht]
%   \center
%   \includegraphics[width=0.6\textwidth]{images/t2k/tpc-eps-converted-to.pdf}
%   \includegraphics[width=0.6\textwidth]{images/t2k/tpc-eps-converted-to.pdf}
%   \caption[Event display of a selected neutrino NC gamma
%   event]{Event display of a selected neutrino NC gamma
%   event. \textbf{\textit{Top:}} \Gls{MC} \Gls{NCg}
%   event. \textbf{\textit{Bottom:}} Data event. The
%   electron~/~positron pair is visible (reconstructed objects in ,
%   true objects in ).}
%   \label{fig:eventdisplay}
% \end{figure}

%%% Local Variables:
%%% mode: latex
%%% TeX-master: "Thesis"
%%% End:
