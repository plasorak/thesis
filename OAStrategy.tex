

In this section, the strategies for oscillation analysis in \Gls{TK}.
There are three main analysis on \Gls{TK} that produce oscillation
parameter result. Two of them use a semi-frequentist approach and
essentially produce confidence intervals, after running a fit over
oscillation parameters and nuisance parameters, these two analysis are
called \gls{p-theta} and \Gls{VALOR}~\cite{VALOR}. The other one uses
a Markov-Chain Monte Carlo (\Gls{MCMC}) to sample over the parameter
space, called \Gls{MaCh3}~\cite{MACH3}. It is a fully bayesian
analysis and produces credible intervals on neutrino mixing parameter.

These analyses use the input of different \Gls{TK} groups. The inputs
are listed here:
\begin{itemize}
\item The beam group provides the absolute flux of histogram (such as
  the one shown in Figure~\ref{fig:flux}), the flux covariance matrix
  which encloses all the on these histograms (see
  Figure~\ref{fig:fluxcov}) and the flux tuning which is, as described
  in Section~\ref{fig:beamline}, gotten from in situ measurement of the
  beam and additional hadron production data.
\item The neutrino interaction working group (\Gls{NIWG}) provides a
  parametrisation for the cross section and prefit systematic errors
  on each of the nuisance parameter of interest. These generally rely
  on the use of external data sets and fit such as the one described
  in Ref.~\cite{CallumFit}, discussions with theorists, and intense
  phenomenological questioning.
\item The \Gls{ND} data, which is used prior to main oscillation fit
  to contrain the flux and cross section systematic
  uncertainties. Traditionally, the \Gls{ND} data that was used for
  the fits was restricted to the \Gls{numu} and \Gls{anumu} data,
  however the aim of this analysis is to unclude the \Gls{nue} data to
  constrain parameter additional cross section and flux parameters.
\item The \Gls{SK} \Gls{CCQE}-like \gls{numu}, \gls{nue}, \gls{anumu}
  and \gls{anumu} samples, along with the \Gls{nue} \Gls{CC}$1\pi^+$
  sample.
\end{itemize}

Note that there is much more data from the \Gls{ND} than the
\Gls{SK}. This means that \Gls{ND} data uncertainty is largely
dominated by systematics, whereas the \Gls{SK} data uncertainty is
mostly statistical (especially for the appearance samples, the
\gls{nue} and \gls{anue}), although this is become less and less true
as \Gls{TK} data is being collected.

For all three analyses, the first step is to only fit the \Gls{ND}
data to constrain parameters. Once an acceptable p-value is reached,
it is considered that the parametrisation is sufficient for an
oscillation fit.

There are other mechanisms to check that the parametrisation is
sufficient under significant change of cross section models, called
fake data, but these are beyond the scope of the analysis that is
presented here.

In this section, the focus is on the \Gls{ND} fits that are used in
frequentist oscillation analysis. This step is essential to reduce the
systematic uncertainties on the cross section and neutrino flux.




%%% Local Variables:
%%% mode: latex
%%% TeX-master: "Thesis"
%%% End:
