
This thesis describes the search for neutrino-production of single
photons using the off-axis near detector at 280 metres (ND280) of the
T2K experiment. A photon selection is used to perform the searches
using the first Fine Grained Detector (FGD1) of the ND280. The thesis
also highlights the importance of systematic uncertainties in the
analysis, since the selection is background dominated. After careful
characterisation of the systematic uncertainties and estimation of the
efficiency, it is concluded that, with the selected \ndataevents data
events and the expected background of \nbkgevents events, the limit
for neutrino-induced single photons, at T2K energies, is
$\dataresult$. This result can be compared with the expected limit of
$\mcresult$. Using ND280’s neutrino energy distribution (peaked at
600~MeV), NEUT predicts a flux-averaged cross section of $\neuttruth$.

A fit to the muon and electron (anti-) neutrinos selections in the
ND280 was performed. The aim of this analysis is to use a data-driven
method to constrain the electron (anti-) neutrinos background events
at SK, the far detector and electron neutrino cross section parameters
for oscillation analyses. These are fundamental inputs in the context
of the searches for Charge-Parity (CP) violation in the neutrino
sector. After a fit to the nominal Monte Carlo was realised, the
electron neutrino and anti-neutrino cross section normalisation
uncertainties are found to be \electronnuerr and \electronanuerr,
repectively. Although these numbers are much higher than the assumed
$3\%$ uncertainty of all the CP violation searches performed at T2K up
to now, the difference in the $\delta_{\text{CP}}$ log-likelihood is
found to be acceptable as the one sigma contours are not very
different and the exclusion of the $\delta_\text{CP} = 0$ is roughly
the same.



