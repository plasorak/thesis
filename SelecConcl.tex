
In this section, the \Gls{NCg} events selection was detailed. The
selection relies on the identification of a photon which decays into a
pair of electron~/~positron in the \Gls{FGD}1. Both these tracks have
to propagate in the \Gls{TPC}2, where a electron \Gls{PID} is
realised. The two tracks system should also be consistent with a
photon. This is realised by requiring that the reconstructed invariant
mass should be small ($<50$~MeV). After this selection, it is found
that the efficiency is $(1.04 \pm 0.07)$~\%. The selection is largely
dominated by backgrounds ($58.86$ events) and the expected number of
\Gls{NCg} events in the selection is $0.14$, for a number of data
events equal to $44$. These numbers allow us to conclude that:
\begin{itemize}[noitemsep,topsep=0pt]
\item The analysis will lead to a limit on the \Gls{NCg} cross section
  rather a \Gls{NCg} cross section,
\item The data limit will be lower than the expected result which
  comes from \Gls{MC}. This is due to the fact that the number of
  observed data event is less that the number of \Gls{MC} events.
\end{itemize}










%%% Local Variables:
%%% mode: latex
%%% TeX-master: "Thesis"
%%% End:
