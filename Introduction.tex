The thesis covers two topics, a search for \nisp and the measurement
of oscillation systematic uncertainties using (anti-) electron
neutrino selections. Both the analyses were done with the Near
Detector at 280 metres (\Gls{ND}) of the Tokai to Kamioka experiment
(\Gls{TK}).

The first chapter covers the introduction to neutrino physics and
includes a brief history of neutrinos and a description of the
neutrino oscillation phenomenon which is being measured at
\Gls{TK}. It also covers the neutrino scattering physics landscape for
\Gls{TK} energies (few 0.5 to a few GeV).

The second chapter describes the \Gls{TK} experiment, consisting of an
accelerator, a near detector complex and a far detector called ``Super
Kamiokande'' (\Gls{SK}).

An additional task of monitoring the data quality of the
electromagnetic calorimeter (\Gls{ECal}) is described in
Chapter~\ref{chap:dataquality}.

Chapter~\ref{chap:pheno} covers the models leading to
neutrino-production of single photon.

The first topic of the thesis, a search for \nisp is presented in
Chapters~\ref{chap:select}, \ref{chap:syst} and \ref{chap:result}.  It
highlights the rationale for conducting the search, methodology for
event selection, evaluation of systematic uncertainties and limit
calculation of the cross section.

Finally, Chapter~\ref{chap:banff} describes the \Gls{ND} fits done to
reduce systematic errors for oscillation analysis, including (anti-)
electron neutrino samples.