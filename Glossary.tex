\newglossaryentry{JPARC}{name=J-PARC, description={Japan Proton
Accelerator Research Complex, the facility that is used to create the
T2K neutrino beam, in Tokai}}

\newglossaryentry{Asimov}{name=Asimov, description={Asimov data set,
the ``best guess'' Monte Carlo prediction}}

\newglossaryentry{CL}{name=CL, description={Confidence Level}}

\newglossaryentry{MIPEM}{name=MIPEM, description={ECal variable used
for discrimination of Minimum Ionising Particle or Electro-Magnetic
object. It is the discriminator after running a boosted decision tree
on electron and muon particle guns}}

\newglossaryentry{EMHIP}{name=EMHIP, description={ECal variable used
for discrimination of Electro-Magnetic object and hadronic shower. It
is the discriminator after running a boosted decision tree on electron
and a proton particle guns}}

\newglossaryentry{SCM}{name=SCM, plural={SCMs}, description={Slave
Clock Module, used in each subdetector of the ND280}}

\newglossaryentry{DPT}{name=DPT, description={Data Processing Task}}

\newglossaryentry{MCM}{name=MCM, description={Master Clock Module,
used in the ND280}}

\newglossaryentry{sand}{name=sand, description={Neutrino events
happening in the sand around the ND280}}

\newglossaryentry{magnet}{name=magnet, description={Neutrino events
happening in the volume enclosed by the magnet in the ND280}}

\newglossaryentry{SiPM}{name=SiPM, description={Silicon
Photo-Mulitplier, on T2K they are MPPCs}}

\newglossaryentry{rdp}{name=rdp, description={real data processing of
the ND280 data}}

\newglossaryentry{pc1}{name=pc1, description={partially calibrated
processing of the ND280 data}}

\newglossaryentry{fpp}{name=fpp, description={first pass processing
(with no calibration) of the ND280 data}}

\newglossaryentry{SNO}{name=SNO, description={Sudbury Neutrino
Observatory}}

\newglossaryentry{PDF}{name=PDF, description={Probability Density
Function, or Parton Distribution Function}}

\newglossaryentry{MR}{name=MR, description={Main Ring at the J-PARC
facility, accelerating protons to 30~GeV}}

\newglossaryentry{SK}{name=SK, description={Super-Kamiokande, a
22~kTon water Cherenkov in Japan, used as the far detector of the T2K
experiment}}

\newglossaryentry{LINAC}{name=LINAC, description={LINear ACcelerator,
the first accelerator which accelerates $H^-$ up to 181~MeV at the
J-PARC}}

\newglossaryentry{INGRID}{name=INGRID, description={Interactive
Neutrino GRID, the neutrino detector at 280~m of the target at the
near side which primarily serves to measure the beam center}}

\newglossaryentry{ND}{name=ND280, description={Near Detector at
280~metres, the off-axis neutrino detector of the target at the near
site, which is used for near detector fits (BANFF), and cross section
measurements}}

\newglossaryentry{RCS}{name=RCS, description={Rapid Cycling
Synchrotron, the second stage of proton acceleration after the LINAC
which accelerates the protons up to 3~GeV}}

\newglossaryentry{ESM}{name=ESM, description={Electro-Static beam
position Monitor, device which uses a capacitor to measure the beam
center of in the secondary beamline}}

\newglossaryentry{OTR}{name=OTR, description={Optical Transition
Radiation, device which measures the beam center centre 280~mm before
it hits the target, it is composed of a fluorescent thin foil which is
monitored by a camera}}

\newglossaryentry{EMFP}{name=EMFP, plural=EMFPs,
description={Effective Mean Free Path, the mean effective distance the
photons traverse before converting in the OOFV regions}}

\newglossaryentry{FHC}{name=FHC, description={Forward Horn Current,
neutrino enhanced beam mode}}

\newglossaryentry{RHC}{name=RHC, description={Reverse Horn Current,
anti-neutrino enhanced beam mode}}

\newglossaryentry{WSF}{name=WSF, description={Wavelength Shifting
Fiber, used in the scintillator detector, it carries the light from
the center of the bar to the MPPC}}

\newglossaryentry{NCEl}{name=NCEl, description={Neutral Current
Elastic}}

\newglossaryentry{ECal}{name=ECal, plural=ECals, description={
Electromagnetic Calorimeter subdetector of the ND280, which aims at
measuring the escaping EM objects from the tracker region}}

\newglossaryentry{PD}{name=P0D, description={Pi-zero [sub]Detector
of the ND280 which aims at measuring neutral pions}}

\newglossaryentry{P0DECal}{name=P0DECal, description={Electromagnetic
Calorimeter surrounding the P0D, which aims at measuring escaping EM
objects from the P0D}}

\newglossaryentry{TPC}{name=TPC, plural=TPCs, description={Time
Projection Chamber, a subdetector of the ND280, which realises precise
PID, charge and momentum measurement for charged particles}}

\newglossaryentry{FGD}{name=FGD, plural=FGDs, description={Fine Grain
Detector, a subdetector of the ND280, which usually is used as target
mass for most analysis}}

\newglossaryentry{BLM}{name=BLM, plural=BLMs, description={Beam Loss
Monitor, device containing gas which detects the charge particles
escaping from the beam pipes at the J-PARC, it is used to stop the
beam when the beam losses exceed a certain value}}

\newglossaryentry{SMRD}{name=SMRD, description={Side Muon Range
[sub]Detectors of the ND280, which aims at measuring the muon ranging
out of the ND280, and serve as cosmic ray muon trigger}}

\newglossaryentry{DsECal}{name=DsECal, description={Downstream ECal,
the most downstream part of the ECal}}

\newglossaryentry{BrECal}{name=BrECal , description={Barrel ECal, side
part of the ECal (around the FGDs and the TPCs)}}

\newglossaryentry{BeRPA}{name=BeRPA, description={Effective RPA, in
the Bernstein polynial basis, the parametrisation used on T2K for
RPA}}

\newglossaryentry{HPD}{name=HPD, description={Highest Posterior
Density. In this thesis, this is a method to assign an error in the
case of asymmetrical errors}}

\newglossaryentry{FV}{name=FV, description={Fiducial Volume of a
particular detector (usually the FGD1 in this thesis)}}

\newglossaryentry{OOFV}{name=OOFV, description={Out Of Fiducial
Volume of a particular detector (usually the FGD1 in this thesis)}}

\newglossaryentry{OOAFV}{name=OOAFV, description={Out Of All the
Fiducial Volumes of all the detectors}}

\newglossaryentry{TK}{name=T2K, description={Tokai to Kamioka,
neutrino oscillation experiment in Japan using the J-PARC beam
(off-axis)}}

\newglossaryentry{SCC}{name=SCC, description={Second Class Currents,
which depends on the mass of the outgoing lepton in neutrino
scattering, hence could be responsible for difference between electron
and muon neutrino cross sections}}

\newglossaryentry{HK}{name=HK, description={Hyper-KamiokaNDE, a
plananed neutrino experiment using an upgraded J-PARC facility and two
50~kTon water Cherenkov detectors as far detectors (sometime loosing
referring to the far detector only), aims at discovering $\Delta CP$
in the neutrino sector}}

\newglossaryentry{SBN}{name=SBN, description={Short Baseline Neutrino
program at Fermilab, composed of the ICARUS, MicroBooNE and SBN
detector, which are on-axis detector in the Booster neutrino beam}}

\newglossaryentry{ADC}{name=ADC, description={Analog to Digital
Converter}}

\newglossaryentry{MINERVA}{name=MINER$\nu$A, description={Fermilab
neutrino scattering experiment, using the NUMI neutrino beam (on-axis)}}

\newglossaryentry{MiniBooNE}{name=MiniBooNE, description={Fermilab
sterile neutrino and neutrino scattering experiment, using the Booster
neutrino beam (on-axis)}}

\newglossaryentry{NOvA}{name=NO$\nu$A, description={NUMI Off-axis
neutrino $\nu_e$ Appearance neutrino experiment, Fermilab experiment
using NUMI neutrino beam (off-axis)}}

\newglossaryentry{SciBooNE}{name=SciBooNE, description={Neutrino
experiment at Fermilab using a scintillator detector in the Booster
neutrino beam (on-axis)}}

\newglossaryentry{ArgoNeuT}{name=ArgoNeuT, description={LArTPC
neutrino experiment at Fermilab}}

\newglossaryentry{LArTPC}{name=LArTPC, description={Liquid argon Time
Projection Chamber}}

\newglossaryentry{NuTEV}{name=NuTEV, description={Fermilab high energy
neutrino scattering experiment, using the Tevatron beam dump}}

\newglossaryentry{NOMAD}{name=NOMAD, description={Neutrino Oscillation
Magnetized Detector, near detector of the OPERA experiment, at CERN}}

\newglossaryentry{K2K} {name=K2K, description={Neutrino oscillation
experiment in Japan, KEK to Kamioka}}

\newglossaryentry{EM} {name=EM , description={Electro-Magnetic}}

\newglossaryentry{MPPC} {name=MPPC ,plural=MPPCs,
description={Multi-Pixel Photon Counter, Photon counter used in the
T2K experiment for all the scintillator detectors at the near site}}

\newglossaryentry{PPO} {name=PPO , description={5-Diphenyloxazole
(wavelength shifter organic scintillator)}}

\newglossaryentry{POPOP}{name=POPOP ,
description={1,4-bis(5-phenyloxazol-2-yl) benzene (wavelength shifter
organic scintillator)}}

\newglossaryentry{TFB} {name=TFB , plural=TFBs, description={Trip-T
Frontend Board, digitaliser for the MPPCs}}

\newglossaryentry{FPGA} {name=FPGA , description={Field-Programmable
Gate Array}}

\newglossaryentry{SSEM} {name=SSEM , description={Segmented Secondary
Emission Profile Monitors, ``comb'' which is placed in the secondary
beam at the J-PARC to measure its profile by collecting the secondary
electrons that are created when the protons interact with it}}

\newglossaryentry{RMM} {name=RMM , plural=RMMs, description={Readout
Merger Module, merger for the TFB signals}}

\newglossaryentry{POT} {name=POT , description={Proton On Target,
measure of the total intensity that the experiment was exposed to}}

\newglossaryentry{PMT} {name=PMT , plural=PMTs,
description={Photo-Multiplier Tube, the device that collect the
Cherenkov light on the walls of Super-Kamiokande}}

\newglossaryentry{PMNS} {name=PMNS , description={Pontecorvo Maki
Nakagawa Sakata, often attached to matrix for neutrino mass mixing}}

\newglossaryentry{MC} {name=MC , description={Monte Carlo
simulations}}

\newglossaryentry{PID} {name=PID , description={Particle
IDentification}}

\newglossaryentry{NEUT} {name=NEUT , description={Neutrino event
generator (for T2K \& SK)}}

\newglossaryentry{GENIE} {name=GENIE , description={General purpose
neutrino event generator}}

\newglossaryentry{highland2}{name=Highland2 , description={HIGH Level
Analysis at ND280, version 2, used for event selection and analysis at
the near detector}}

\newglossaryentry{psyche} {name=psyche , description={Propagation of
SYstematics and CHaracterization of Events, comes with highland2,
applies all the detector systematic uncertainties to the event
selections}}

\newglossaryentry{BANFF}{name=BANFF , description={Beam And ND280 Flux
extrapolation task Force, the fitting framework that realise the near
detector fit and constrain the cross section and flux systematic
uncertainties before every oscillation analyses using the
Super-Kamiokande data}}

\newglossaryentry{T2KReWeight}{name=T2KReWeight , description={A
software package developped in T2K to change the cross section of an
event via its weight according to ``dials'' (fundamental inputs to the
calculations of the cross section)}}

\newglossaryentry{JReWeight}{name=JReWeight , description={A software
package developped in T2K to propagate the flux uncertainty. This flux
uncertainty is traditionnally propagated by modify the relative
importance of neutrino events according to their energy.}}

\newglossaryentry{QE} {name=QE , description={Quasi-Elastic}}

\newglossaryentry{CC} {name=CC , description={Charged Current}}

\newglossaryentry{CCQE} {name=CCQE ,description={Charged Current Quasi
Elastic}}

\newglossaryentry{NC} {name=NC , description={Neutral Current}}

\newglossaryentry{NCg} {name=NC$\gamma$, description={Neutral Current
single photon}}

\newglossaryentry{RES} {name=RES , description={Resonant, a process
where a boson interacts with a nucleon and creates a nuclear
resonance}}

\newglossaryentry{MIP} {name=MIP , description={Minimum Ionising
Particle, the minimum energy a particle looses by unit distance (for a
muon, this is typically $2 \text{MeV}/\text{cm}$ in a matterial that
has a density of $1 \text{~g}/\text{cm}^3$)}}

\newglossaryentry{DIS} {name=DIS , description={Deep Inelastic
Scattering, a process where a boson interacts with a single quark}}

\newglossaryentry{SIS} {name=SIS , description={Shallow Inelastic
Scattering (also refered as ``Transition region''), an intermediate
process between resonant and DIS, which has no classical
interpretation}}

\newglossaryentry{MT} {name=MT , description={Main Track, the highest
momemtum track with positive of negative charge, propagating from the
FGD1 to the TPC2}}

\newglossaryentry{PT} {name=PT , description={Pair Track, the second
highest momentum track with an opposite charge to the one of the MT,
propagating from the FGD1 to the TPC2}}

\newglossaryentry{COH} {name=COH , description={Coherent, a process
where the boson interacts with the whole nucleus rather than single
nucleon, sometimes defined as a process which leaves the nucleus in
its ground state}}

\newglossaryentry{MEC} {name=MEC , description={Meson Exchange
Current, a process where a boson interacts with a pair of correlated
nucleons in the nucleus (thus exchanging a pion in the chiral theory).
Most often on T2K, MEC refers to any correction that has to be added
to a standard CCQE in a nucleus (except the RPA correction), and no
difference is made with 2p2h events.}}

\newglossaryentry{2p2h} {name=2p2h , description={2 particles 2 holes
nucleus excitation, a process where a boson interacts with a nucleon
or a pair of nucleons, where, because of FSI, correlation or exchange
of pion excitation, two nucleon get excited}}

\newglossaryentry{RFG} {name=RFG , description={Relativistic Fermi
Gas, a parametrisation of the density of states in the nucleus, which
only depends on the number of nucleons in the nucleus}}

\newglossaryentry{LFG} {name=LFG , description={Local Fermi Gas, a
parametrisation of the density of states in the nucleus, which depends
on the local density in the nucleus}}

\newglossaryentry{SF} {name=SF , description={Spectral Function, a
parametrisation of density of states in the nucleus, taking into acount
the interaction between the nucleons}}

\newglossaryentry{FSI} {name=FSI , description={Final State
Interactions, the interaction which a pion or a nucleon undergo before
exiting the nucleus (i.e. sometime a charged pion created inside the
nucleus will be reabsorbed by the nucleus and will not be visible in
the detector)}}

\newglossaryentry{CP} {name=CP , description={Charge Parity}}

\newglossaryentry{numu} {name=\numu , description={muon neutrino}}

\newglossaryentry{PEU} {name=PEU , description={Pixel Equivalent Unit,
pC value detected in a detector}}

\newglossaryentry{NUISANCE}{name=NUISANCE , description={NeUtrino
Interaction Systematics ANalyser by Comparing Experiments, NeUtrino
Interaction Synthesiser Aggregating Constraints from Experiments, or
NeUtrino Interaction Systematics from A-Neutrino sCattering
Experiments, ...}}


\newglossaryentry{NUANCE} {name=NUANCE , description={Neutrino
interaction generator used in MiniBooNE}}

\newglossaryentry{nue} {name=\nue , description={electron neutrino}}

\newglossaryentry{anumu} {name=\anumu , description={muon
anti-neutrino}}

\newglossaryentry{anue} {name=\anue , description={electron
anti-neutrino}}

\newglossaryentry{piz} {name=\piz , description={neutral pion}}

\newglossaryentry{pipm} {name=\pipm , description={charged pion}}

\newglossaryentry{p} {name=p, description={proton}}

\newglossaryentry{n} {name=n , description={neutron}}

\newglossaryentry{CT} {name=CT , description={Current Transformer, a
device in the secondary beam line at J-PARC which measures the
intensity of the proton beam by measuring the induced current on a
coil around the beam pipe}}

\newglossaryentry{RMS} {name=RMS , description={Root Mean Square, a
statistical quantity related to the spread of ensemble of number}}

\newglossaryentry{RPA} {name=RPA ,description={Random Phase
Approximation, a boson screening effect which usually depends on the
$Q^2$ of the reaction. At low $Q^2$, the RPA correction leads to a
damping of the cross section; at medium $Q^2$, the correction is an
enhancement of the cross section; at high $Q^2$, the correction is 1}}

\newglossaryentry{p-theta} {name=p-theta , description={Neutrino
oscillation parameter fitting software taking into account the p-theta
distribution of the electron neutrino appearance signal, used for T2K
oscillation analyses}}

\newglossaryentry{VALOR} {name=VaLOR
,description={Valencia-Lancaster-Oxford-RAL neutrino oscillation
parameters fitting software}}

\newglossaryentry{MaCh3} {name=MaCh3, description={Markov-Chain 3
flavours neutrino oscillation parameter fitting software, used for T2K
oscillation analyses, relying on the use of a MCMC method}}

\newglossaryentry{MCMC} {name=MCMC ,description={Markov-Chain Monte
Carlo, a method which is used by MaCh3 to sample the allowed parameter
space and get a posterior likelihood distribution given observed data
and a parametrisation}}

\newglossaryentry{NIWG} {name=NIWG , description={Neutrino Interaction
Working Group, the group in T2K which creates the cross section
parametrisaton and assign errors to it}}

\newglossaryentry{DUNE} {name=DUNE , description={Deep Undergroud
Neutrino Experiment, an planned neutrino experiment in Fermilab and
Sanford Underground Research Facility aiming to discover $\Delta CP$
in the neutrino sector}}

\newglossaryentry{CHORUS} {name=CHORUS ,description={CERN Hybrid
Oscillation Research ApparatUS}}

\newglossaryentry{CDHSW} {name=CDHSW ,
description={CERN-Dortmund-Heidelberg-Saclay-Warsaw experiment}}

\newglossaryentry{EMC} {name=EMC , description={European Muon
Collaboration, the name of the experiment that discovered the ``EMC
effect,'' which is an unexpected reduction of the electron DIS cross
section for nuclear targets as a function of the Bjorken $x$ variable
in the high range ($x > 0.3$)}}

\newglossaryentry{anti-shadowing} {name=anti-shadowing,
description={Enhancement of the electron DIS} cross section for
nuclear targets as a function of the Bjorken $x$ variable in the
middle range ($0.05<x<0.2$)}

\newglossaryentry{HERMES} {name=HERMES , description={e-p experiment
at the HERA collider}}

